\documentclass[12pt]{article}
\usepackage{myarticle,switcheml,mflogo,garamond}

\title{Automatic Greek Font Generation}
\author{Charles Duan \\ \switchemail{cduan@fas.harvard.edu}}

\begin{document}
\maketitle

\section{Introduction}

The advent of computer technology has led to significant advances in the world
of typography. Rather than being fixed to the typographical conventions of a
publishing house or a typewriter, an author has the freedom of formatting a
document in the way that author prefers. In particular, computerized typesetting
engines and word processors such as \TeX\ and Microsoft Word allow the creator
of a document to choose the typeface, or font, of the document, and the ease of
computer-aided design has produced myriad fonts from which the author may
choose.

Unfortunately, these computer-designed fonts are almost exclusively created for
an English or Romance-language world. Practically every font available contains
the uppercase and lowercase of the Roman alphabet---and practically no font
contains any other alphabet.

This is not problematic for documents \emph{entirely} in other languages. A
document in Chinese will simply choose from a restricted but complete set of
Chinese fonts. However, consider the case in which \emph{both} a Roman alphabet
and another are used in the same document. This is a common case in mathematics
typesetting, where the majority of the text uses the Roman alphabet but Greek
letters and symbols are interspersed throughout. In these cases, the author
might wish to choose the font for the Roman text out of the many fonts
available, but what font is to be used for the Greek text and the symbols?

The first option is for the author to choose \emph{only} from fonts that have a
companion Greek set. The Computer Modern font, developed by Donald Knuth,
includes one; other fonts such as Lucida Bright and Times New Roman have them as
well. A second solution is to simply choose an unrelated font for the Greek
alphabet and symbols. All too often papers today are published using, say, the
Computer Modern fonts for the symbols and Greek letters, and another font for
the Roman text. However, if the fonts are \emph{visually incompatible}---the
thickness, height, and features of the letters differ---the result can be
displeasing and distracting to the reader. The sensitivity of the human eye can
detect even minor incompatibilities.

An ideal solution would be to take a given Roman text font, perform some
analysis on it, and automatically generate a Greek font that is visually
compatible with the Roman font. There are two requirements necessary to
implement this solution: an abstract template for generating the Greek and
symbol fonts, and a mechanism for capturing the salient features of the Roman
font so that those features may be transfered to the Greek and symbol
letterforms.

The first requirement, an abstract template for the Greek and symbol fonts, in
fact already exists. The Computer Modern fonts are in fact a set of highly
parametrized \emph{programs} that can generate a variety of different fonts. The
programs are controlled by 62 parameters defining the height of various letters,
the thickness of stems and curves, and other features of the letters. What
remains, then, is the development of a program to automatically calculate, based
on the existing Roman font, the appropriate parameters to generate a visually
compatible set of Greek and symbol fonts.

\section{Overview of the Algorithm}

Of the 62 parameters to the Computer Modern font programs, nine of them define
broad features of the font, such as whether the font should be sans serif. The
remaining parameters are numeric dimensions of various features of the letters.
The first task of the automated font generation system, then, is to use the
outline descriptions of the Roman letterforms to take the appropriate
measurements.

Once those raw measurements are taken, they may be processed to form the final
set of parameters. In some cases, several measurements may be taken for the same
parameter, and the values will be averaged. Other parameters may be calculated
based on several of the measurements. Once the processing of the parameters is
complete, the final set of parameters is produced, and the \MF\ program can be
run to interpret the Computer Modern font programs and produce the new, visually
compatible Greek and symbol fonts.

The produced fonts are in a rasterized bitmap form, which does not display well
on computer screens. To make the fonts optimally usable, they are run through an
outline tracing program to generate outline fonts. This is a well-known
procedure; the outline versions of the Computer Modern fonts were developed in
this way.

Finally, we must make the fonts usable to the typesetting system. We make the
fonts usable to \LaTeX\ because the output of \MF\ is particularly suitable to
\LaTeX, and it is also very easy to automatically test the appearance of
mathematics typesetting with it. The FontInst program is used to install the
fonts so that they can be used just as easily as any other font.

\section{The Choice of Technologies}

Two key design choices in developing this system were the choice to use the
Adobe Type 1 format and the PostScript language to perform the parameter
measurements and the choice to use \MF\ and the Computer Modern programs to
produce the new fonts. Each of these choices has advantages and disadvantages,
and it is worthwhile exploring them and considering alternatives.

The primary reason for using Type 1 fonts is simply that \TeX\ and its
associated programs can support it best. Other outline font formats, such as
TrueType, are commonly used in \TeX\ by first converting them to Type 1.
Additionally, using an outline format makes it very easy to deal with the
boundaries of letters, so it is easy to find, for instance, intersections
between the outline of a letter and a straight line. The measurement programs
are written in PostScript because that is the native language of the Type 1
format.

One difficulty of implementing the measurement programs in PostScript is that a
great deal of numerical instability arises. For example, a common procedure in
the measurements is to find the intersection of two straight line segments. This
is easy to implement, but if one of the line segments is very short, the
calculations may report that they do not intersect even if they do, because of
rounding errors in the floating-point calculations. As a result, the programs
need to be very carefully written to deal with these rounding errors. However,
most of the checks to deal with these problems can be tucked away in library
routines, so the actual measurement programs do not have to deal with them.

The use of \MF\ and the Computer Modern fonts is generally because they are the
only commonly used parametrizable font system readily available. Additionally,
since the Greek and symbol fonts are already programmed, there is no need to
write an entirely new set of font generation programs to create the new
characters.

One issue with using \MF\ is the fact that it outputs rasterized bitmap fonts
rather than outline fonts. This is unfortunate given that, internally, \MF\
deals with the same Bezier curves and lines that an outline font would use. One
solution would be to use \MP, an interpreter for \MF\ programs that produces
PostScript output. In fact, there is a program (mf2pt1) that can execute \MP\
and then put together the Type 1 font from its output. However, mf2pt1 is not
sufficiently mature for current use (it cannot handle all of the drawing
capabilities of \MF\ or \MP).\footnote{There is a mature program for this
purpose, MetaFog, but it is only commercially available.} Additionally, there
are certain drawing operations employed by \MF\ that \emph{cannot} be reasonably
transferred to a Type 1 font, such as the ability to erase parts of curves. As a
result, it is necessary to use an outline tracing utility to produce an outline
font from the raster. Luckily, the shapes of the letterforms are simple enough,
so a tracing utility can do a very acceptable job.

Another possible way of generating Greek and symbol fonts from the Roman set is
to devise some sort of modifications to the Roman letters to obtain the Greek
and symbol ones. For certain Greek letters this is a simple task: the Greek
capital letters Gamma and Lambda ($\Gamma$, $\Lambda$) can be obtained by
removing the middle horizontal strokes of the Roman letters F and A,
respectively. This would be an optimal solution since it would not only capture
not only the measurable aspects of the letterforms but also the artistic design,
such as the calligraphic effect at the top of the capital A in Adobe Garamond.
However, other letters and symbols (e.g., $\pi$ and $\wp$) would be much harder
to form by some modification to Roman letters.

\section{The Measurement Libraries}

In order to perform measurements on the Roman letterforms, we first need a
library of routines that give sufficient functionality to make any sort of
general measurement on a letter. In this section we describe this library of
PostScript routines that can be used to perform these operations.

In the PostScript language, the interpreter maintains a ``current path,'' a set
of line segments and Bezier curves that define a path through a two-dimensional
plane. It is simple to produce a path that corresponds to the outline of a given
letter (using the \emph{charpath} operator), and the PostScript language defines
several low-level routines that can be used to traverse this path and operate on
it. This ability to produce the outline path for a character and then operate on
it forms the basis for our system.

PostScript maintains a stack on which numbers and other objects may be placed;
arguments to functions are taken from the stack. In the following routines, a
\emph{point} refers to two numbers on the PostScript stack, representing the $x$
and $y$ coordinates of the point. A \emph{line} refers to two points (four
numbers) on the stack. An \emph{angle} is a single number on the stack
representing the number of degrees from the horizontal. A \emph{curve} is four
points (8 numbers) on the stack, representing the start point of a Bezier curve,
the two control points, and the endpoint. An \emph{array of points} is an array
for which each element is a two-element array; the inner arrays represent
points.

\subsection{Low-Level Routines}

These routines perform simple operations. They generally operate on single
points or numbers.

\begin{description}
\item[withinRange] Given three numbers, a value $n$, a base $b$, and a range
$r$, determines whether $n$ is between $b$ and $b+r$.

\item[revArray] Given an array, reverses it.

\item[loadArray] Given an array, places its contents on the stack and removes
the array.

\item[firstElt, lastElt] Given an array, returns its first or last element.

\item[avgPoint, addPoint, subPoint, dotProduct] Given two points, computes their
average, sum, difference, or dot product.

\item[eqPoint] Given two points and a tolerance, calculates the points'
approximate equality.

\item[scalePoint] Given a point and a number, multiplies each of the coordinates
of the point by the number.

\item[magnitude] Given a point, determines its magnitude from $(0,0)$.

\item[xcoor, ycoor] Given a point, returns one coordinate.

\item[defPoint] Given a point and a name (or a name and a point), defines
\emph{name} to return that point.

\item[uniquePoints] Given an array of points, discards points that are too close
to each other and returns a list of ``unique'' points.

\item[sortByX, sortByY] Given an array of points, sorts that array by the
points' $x$ or $y$ coordinates.

\item[min, max] Given two numbers, returns the minimum or maximum.

\item[seppairs] Given values $x_1, y_1, x_2, y_2,\ldots,x_n,y_n,n$, replaces
them on the stack with $x_1,\ldots,x_n,y_1,\ldots,y_n$.

\end{description}

\subsection{Line and Curve Routines}

These routines perform basic operations on lines and curves, the components of
paths.

\begin{description}
\item[curvetoPoint] Requires a Bezier function and a number $t$ between 0 and 1.
A Bezier curve can be parametrized as follows:
\[
(x,y) = (x_0,y_0)(1-t)^3 + 3(x_1,y_1)t(1-t)^2 + 3(x_2,y_2)t^2(1-t) +
(x_3,y_3)t^3
\]
where $(x_0,y_0)$ and $(x_3,y_3)$ are the endpoints and $(x_1,y_1)$ and
$(x_2,y_2)$ are the control points. This function calculates $(x,y)$.

\item[lineAngle] Given a line, determine the angle between that line and the
horizontal, in degrees.

\item[curveStartAngle, curveEndAngle] Given a curve, determine the angle that
the curve is traveling at the start or end of the curve.

\item[intersect] Given two lines, returns the intersection point (if any) and
whether or not they intersected.

\item[linePtIntc] Given a line and a point, determines whether or not the point
is close to the line. If so, then the point on the line closest to that point is
returned.

\item[curvePtIntc] Given a line and a curve, determines whether or not the point
is close to the curve.

\item[curvetoExtremes] Given a curve, finds the horizontal and vertical extrema
of that curve.

\item[curvetoIntc] Given a curve and a line, finds the intersections between
them.

\end{description}

\subsection{Path Functions}

\begin{description}
\item[tracePath] A low-level function that allows for performing a function on
each part of a path. It should be executed in an environment containing
appropriate hooks for curves and lines. This function is generally never called;
the other path functions rely on it though.

\item[pathLineIntc] Given a line, find the points of intersection between it and
the current path.

\item[pathExtremes] Finds \emph{all} of the local extrema for the current path.
Most of the points are not useful, but the global extrema can be identified
using the \emph{sortByX} and \emph{sortByY} routines.

\item[pathMidpoint, pathUR, pathLL] Finds the midpoint, upper right, or lower
left of the bounding box of the current path.

\item[vertSplitLine, horizSplitLine] Given a number between 0 and 1, produces a
line that is horizontal or vertical and splits the current path's bounding box
by that fraction.

\item[vertLineThrough, horizLineThrough] Given a point, draws a horizontal or
vertical line through that point.

\item[letterPath] Given a string consisting of a letter, sets the current path
to that letter.

\item[pathAngle] Given a point on the path, returns the angle that the path is
traveling at that point.

\item[measureHeight, measureDepth] Measure the height or depth of the current
path.

\end{description}

\subsection{Subpaths}

Often it is useful to subset the current path and make inferences on that
subset. The following routines allow for defining a subset of the path based on
certain conditions and then using it for analysis.

\begin{description}
\item[traceSubpath] Requires a procedure that sets up the conditions for
starting and ending the subpath. Executes that procedure, and then travels
through the path, collecting those segments that fall within the subpath
conditions. Returns a ``user path,'' a PostScript object that represents (and
may be used as) a path.

\item[subpathEndpoint] Returns the endpoint of a subpath.

\item[useSubpath] Given a user path and a procedure, temporarily sets the
current path to the user path and executes the procedure under that condition.

\end{description}

The following routines are useful inside the procedure to \emph{traceSubpath},
as they define start and end conditions.

\begin{description}
\item[startWhenAngle] Requires a procedure that returns a boolean value. Begin
the subpath when the angle of the path angle causes that procedure to return
true. Since the path can start anywhere along the edge of the letterform, this
condition is really only useful for creating a ``sub-subpath.''

\item[endWhenAngle] Requires a procedure that returns a boolean value. End the
subpath at some point immediately \emph{after} the path angle causes that
procedure to return true.

\item[endBeforeAngle] Similar to \emph{endWhenAngle}, but the subpath terminates
immediately \emph{before} the procedure returns true.

\item[startAtPoint] Given a point, start the subpath at that point. An attempt
is made to truncate the subpath to start at that point exactly, but the current
implementation does not attempt to divide Bezier curves into parts to do so.

\item[endAtPoint] Given a point, end the subpath at that point.
\end{description}

\subsection{Drawing Routines}

These functions draw on the PostScript page but have no use in terms of
calculations.

\begin{description}

\item[drawSubpath] Given a subpath draw it with a thick line. The subpath is
copied off of the stack, so it remains there after.

\item[pointbox, uppointbox, rtpointbox, ltpointbox] Given a point, draws a
triangle revealing that point. (The name ``point box'' refers to the fact that
they were once drawn as squares around the point.)

\item[drawLine] Given a line, draws it. The line is copied off of the stack.

\item[stringOfText] Given a string and a number, outputs the string, the number,
and the current parameter being measured to the console and to the PostScript
page. This function is always the last one called in any parameter measurement,
as it outputs the measured value.
\end{description}

Note that \emph{letterPath} in fact draws the path of the letter when it is
called.

\section{Examples of Measurement Routines}

In order to understand how a parameter might be measured from a letter, it is
worth considering in detail a few of the measurement routines. In this section
we present routines of varying complexity.

\subsection{A Trivial Measurement}

The parameter \emph{x\_height} measures the height of the lowercase ``x.''

\begin{quote}
\begin{verbatim}
(x_height) measureHeight stringOfText
\end{verbatim}
\end{quote}

\subsection{A Simple Measurement}

The parameter \emph{curve} measures the breadth of wide curve sections such as
the sides of the letter ``o.''

We first ``flatten'' the path, turning curved segments into many small line
segments that accurately describe the curve.
\begin{quote}
\begin{verbatim}
flattenpath
\end{verbatim}
\end{quote}
We then find the extreme points of the path of the ``o'' and identify the
minimum $x$-value point by sorting by $x$ coordinate and then taking the first
element. We must execute \emph{loadArray} because arrays of points store points
as two-element arrays, but points are otherwise represented as just two values
on the stack.
\begin{quote}
\begin{verbatim}
pathExtremes sortByX firstElt loadArray
\end{verbatim}
\end{quote}
Now, we draw a horizontal line through that point and find its intersections
with the outline of the ``o.'' We take the first two elements and name them.
Note that the first element, in all likelihood, is the same as the previous
point we found. Finally, we draw arrows to denote those points. Note that the
\emph{getinterval} operator takes an array, a starting element, and a range, and
returns the subarray corresponding to that offset and range.
\begin{quote}
\begin{verbatim}
horizLineThrough pathLineIntc sortByX
0 2 getinterval loadArray
/oInside defPoint /oOutside defPoint
oInside pointbox oOutside pointbox
\end{verbatim}
\end{quote}
Finally, the parameter is calculated as the difference between the $x$
coordinates.
\begin{quote}
\begin{verbatim}
(Lowercase curve)
oInside xcoor oOutside xcoor sub
stringOfText
\end{verbatim}
\end{quote}

\subsection{A Complicated Measurement}

The \emph{slab} parameter represents the thickness of serifs. We measure the
serif thickness on the lower right serif of a lowercase ``l.''

We first identify some point at the bottom of the ``l'' and some point at the
right of the ``l.'' We then trace the subpath between those two points.
\begin{quote}
\begin{verbatim}
0.5 vertSplitLine pathLineIntc
sortByY firstElt /lBottomPoint defPoint
0.5 horizSplitLine pathLineIntc
sortByX lastElt /lRightPoint defPoint
lBottomPoint pointbox
lRightPoint rtpointbox
{
    lBottomPoint startAtPoint
    lRightPoint endAtPoint
} traceSubpath
\end{verbatim}
\end{quote}
We then trace the sub-subpath that starts when the angle is between
45\textdegree\ and 135\textdegree\ and ends when the angle is between
135\textdegree\ and 225\textdegree. This subpath will thus begin somewhere as
the serif begins to rise and end somewhere when the serif edge becomes
horizontal again. The endpoint of this subpath, then, will be approximately at
the top corner of the serif. We then name and draw the highest point on this
subpath, the ``tip'' of the serif.
\begin{quote}
\begin{verbatim}
{ {
    flattenpath
    { 45 90 withinRange } startWhenAngle
    { 135 90 withinRange } endBeforeAngle
  } traceSubpath
} useSubpath
drawSubpath
{ pathExtremes } useSubpath
sortByY lastElt /lTopSlab defPoint
lTopSlab uppointbox
\end{verbatim}
\end{quote}
Since the serif may be very curved, the thickness of the tip may not be a good
representation of how thick the serif should be. We choose a point about 4/5 of
the way between the stem and the serif tip. We then draw a vertical line through
that point and find the lowest two intersections (those with the smallest $y$
coordinate).
\begin{quote}
\begin{verbatim}
lTopSlab 0.8 scalePoint
lRightPoint 0.2 scalePoint addPoint
vertLineThrough pathLineIntc
sortByY 0 2 getinterval loadArray
/lSlabTop defPoint /lSlabBot defPoint
lSlabTop uppointbox lSlabBot pointbox
\end{verbatim}
\end{quote}
Finally, the \emph{slab} parameter is the difference in $y$ coordinate between
those two points.
\begin{quote}
\begin{verbatim}
(Serif thickness)
lSlabTop ycoor
lSlabBot ycoor sub
stringOfText
\end{verbatim}
\end{quote}

\section{Difficulties in Measurement}

Other than the numerical instabilities, there have been several difficulties in
the measurement techniques. In this section we discuss those difficulties and
how they were dealt with or solved.

\subsection{Multiple Possibilities for Measurement}

For some letters, the choice of which parameter to measure is nontrivial, as
there are often several slightly competing possibilities. The overarching
example is that sometimes the parameters differ for uppercase and lowercase
letters. In some cases, the Computer Modern parametrizations allow for this, by
providing for two parameters, one for uppercase letters and one for lowercase
letters (e.g., \emph{cap\_stem} and \emph{stem} for uppercase and lowercase stem
breadth). In other cases, such as the thickness of hairlines, the uppercase and
lowercase letters share the same parameter. To deal with these situations, we
simply create two parameter sets, one for uppercase and one for lowercase
letters; the uppercase set is used for generating uppercase Greek letters and
the lowercase set for lowercase Greek letters and symbols.

In other cases the problem is more difficult. For example, the thickness of
horizontal bars in uppercase letters is represented by one parameter in Computer
Modern, \emph{cap\_bar}, and it governs the thickness of bars in the letters
``T,'' ``E,'' and ``H.'' It is equally simple to measure the parameter from any
of them, so which should be used? Inspecting the actual use of the
\emph{cap\_bar} parameter reveals that it is only used in the uppercase Pi
character ($\Pi$) and other similar symbols, so the thickness of the capital
``T'' bar seems most appropriate.

As a second example, the \emph{hair} parameter governs the thickness of
``hairline'' strokes that travel vertically. It turns out that this parameter in
the Computer Modern programs is used in only two places in the lowercase Roman
alphabet: first, it determines the ``tails'' of certain letters such as the
lowercase ``t,'' and second, it governs the thickness of thin diagonal stems in
the lowercase ``v,'' ``w,'' ``x,'' and ``y.'' In Computer Modern font those two
thicknesses are identical, but in many other fonts they are substantially
different. Again, the choice of which to use can be determined by looking at
their usage in the Greek fonts: the \emph{hair} parameter determines the
thickness of the thin join areas of the pi character ($\pi$) and the breadth of
the vertical stroke through the phi character ($\phi$), so the thin stroke of
the letter ``v'' seems to be a more accurate representation of the thickness of
the relevant Greek characters.

\subsection{Parameters Measured in Ways Different From Their Definitions}

In some cases, a parameter may be defined in one way and measured with a
completely unrelated algorithm. Consider the \emph{bracket} parameter. Knuth
defines the parameter as follows:
\begin{quote}
\emph{Bracket} is the vertical distance from serif to stem tangent.
\end{quote}
This means that the appropriate way to measure the \emph{bracket} parameter is
to inspect the curve until it becomes tangent to the horizontal (or vertical);
the height of that tangency point is the bracket.

\begin{figure}
\centering
\Huge
\texttt{I}\quad I
\caption{An uppercase ``I'' in Courier and Adobe Garamond. The Courier letter
has no bracket, so the serif meets the stem at a right angle. The Garamond
letter has a large bracket, so there is a wide curve joining the stem to the
serif.}
\end{figure}

This form of measurement will yield poor results for two reasons. First, in some
fonts, the angle may never go quite vertical or horizontal, and the exact point
at which it \emph{does} become vertical may vary greatly from letter to letter,
simply because of minor numerical differences. Second, the feature that
\emph{bracket} represents is not actually the height of the tangency point but
rather the ``darkness'' or amount of curvature between the serif and the stem. A
font like Courier has no bracket at all, so serifs intersect stems at angles In
contrast, Adobe Garamond has a high bracket, so there is a wide curve from the
serif to the stem. The \emph{bracket} parameter should reflect the amount of
curvature that is seen rather than strictly how high the tangency point occurs.

Our bracket height measurement routine works as follows. First, the line
parallel to the stem is intersected with a line parallel to the serif's edge,
identifying the ``corner'' of the serif. Then a diagonal line is drawn extending
outward from the corner until that line intersects with the letter. Based on the
distance between the corner and the intersection point with the letter we can
calculate a reasonable value for \emph{bracket} such that Computer Modern will
draw a similar-looking curve.


\subsection{Reasonable Parameters Yield Unreasonable Letterforms}

In some cases, a parameter that produces high quality results for most
characters produces very strange results for one or two unique characters. The
most striking example of this is in the terminals of the uppercase Lambda
($\Lambda$).

The \emph{bracket} parameter, described above, produces very acceptable looking
serifs for every uppercase Greek letter except for the uppercase Lambda
($\Lambda$). In the case of Lambda, Computer Modern makes two assumptions:
\begin{enumerate}
\item The lower left stem is very thin, so the bracket needs to be increased
considerably to ``thicken'' the lower left corner to give better visual balance.
\item The \emph{bracket} parameter is very small, and because the stem meets the
serif at a diagonal angle, the thickness of the curvature between the stem and
the serif needs to be increased for the acute angle side, or else it will appear
as if there is no bracket at all.
\end{enumerate}
These assumptions work well for Computer Modern because the thin strokes
\emph{are} very thin and the bracket \emph{is} very low. In a font like Adobe
Garamond, though, this assumption fails. The bracket ends up consuming about
half of the height of the stem, and the serif looks more like a blob of
misprinted ink than a serif.

To correct for this, the program generating the Lambda character is rewritten.
First, the bracket of the uppercase ``A'' is measured and used as the bracket
parameter for the Lambda character, and a special ``darkness'' parameter is
measured for the inside acute angle so that the inside of the capital ``A'' and
the inside of the Lambda character have the same curvature.

In general the Computer Modern programs produce very acceptable-looking Greek
and symbol fonts. In some cases, though, the font programs make assumptions
about the parameters that fail to hold; in those cases the characters must be
considered on a case-by-case basis and rewritten appropriately.


\section{Working Examples}

The font generation system is currently able to produce math fonts that may be
used in \TeX. The following pages provide examples of the system using the
following fonts:
\begin{itemize}
\item Times New Roman
\item Palatino
\item New Century Schoolbook
\item Adobe Garamond
\item Courier
\item Computer Modern
\end{itemize}
All of the samples were automatically generated with no hand modifications to
the parameters.

In general, the Greek letters appear fairly pleasing and compatible with the
Roman. In a number of cases, the letters appear a bit thin; in the case of the
$\beta$ and $\gamma$ characters this is especially troubling. However, this
seems to result from the fact that the original designs were a bit too thin as
well. (Note that, on the Courier sample, the letters may appear uniformly too
thin. This is probably because the version of Courier that was measured, the one
included with GhostScript, is different from the version of Courier built into
Microsoft Windows.) The uppercase letters are very acceptable, especially after
the hand-tuning of the programs.

The math symbols are sometimes acceptable and sometimes not so. For example, the
summation symbol ($\sum$) appears for certain fonts to lack the same contrast
between thick and thin strokes that the corresponding Greek letter ($\Sigma$)
displays. Looking at the program for the character reveals why this is so: the
thickness of the lower left stem and the arms are based on the
\emph{rule\_thickness} parameter (which determines the thickness of the bars in
the $+$ and $=$ symbols), while it should really be based on the
\emph{cap\_hair} parameter. There are two ways to repair this: first, constrain
\emph{rule\_thickness} to be approximately \emph{cap\_hair}; second, rewrite all
of the relevant \MF\ programs.

The final concern is that the spacing is often egregiously wrong. The radicals
fail to abut with the bar, as do the equal signs with the arrows to make long
arrows, and many of the letters are too close together or too far apart. A good
deal more work needs to be done in order to repair these spacing errors
appropriately. This is especially obvious in the Courier-based font.

The comparison of the Computer Modern samples deserves special attention. There
is very little, in fact, that is obviously different between the two other than
spacing errors. The only substantially observable difference to the author,
viewing the documents enlarged at 1600\%, is that the tips of sharp cutoffs in
the lowercase Greek letters is sharper in the generated fonts than the
originals, and some hairlines are very slightly lighter in the generated fonts.
These differences are not surprising, as the math italic fonts are in fact
slightly darker than the text fonts (in Donald Knuth's words, ``because
mathematics is important to mathematicians), and the italic fonts of Computer
Modern have rounded edges while the upright fonts (from which the parameters are
measured) have sharp edges. These seem to be more stylistic variations, though,
and they do not greatly affect the compatibility of the generated math fonts
with the text fonts.

\section{Future Work}

The first task that needs to be completed for this system to be immediately
usable is to fix the spacing errors. Once those are fixed, there are several
refinements to the typesetting to be pursued.

First, the programs currently only work for very standard-looking Roman fonts
with serifs. Sans-serif font support would obviously be of great use, and it
should not be difficult to implement given that the Computer Modern programs
can produce a sans-serif variant. Additionally, other less traditional fonts
could be supported to some degree.

The precise spacing between letters is another area of work. Spacing between
characters in math formulae is somewhat different from spacing between
characters in text, and an automated system to arrange the spacing would improve
the quality of the mathematics typesetting.

Third, a number of the Computer Modern programs could use some improvement, as
mentioned above in the discussion of the summation signs. Systematically
improving the character shapes would be a useful line of work in this project.

Fourth, while \MF\ can generate many of these characters very reasonably, some
of them, such as the uppercase Greek letters, can be generated best by
performing modifications to the existing Roman letters. For example, a capital
Lambda ($\Lambda$) could be formed by eliminating the crossbar from a capital
``A,'' and a capital Theta ($\Theta$) can be formed by adding a small internal
crossbar to an uppercase ``O.'' Fonts generated in this way should be the most
compatible with the original Roman font.

\end{document}
