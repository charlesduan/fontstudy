\documentclass[12pt]{article}
\usepackage{mathzmn,times}

\title{A Random Sample of Mathematical~Typesetting}
\author{Charles Duan}

\begin{document}

\maketitle

Let $\alpha$ be a variable such that $\alpha\ge\alpha$ and $\alpha\le\alpha$.
There exists some $\beta$ such that either $\alpha=\beta$ or $\alpha\ne\beta$,
that is:
\[
\forall\alpha\:\exists\beta: \alpha=\beta\in\alpha\ne\beta, x<y\vee x>y
\]
Consider vectors $\vec\nu=(\alpha,\ldots,\beta)$ and $\vec v=\nu\times\nu$. We
wish to find some value $\Lambda$ such that:
\[
\Lambda = \pm \pi \int_{0}^{\infty} \nu\cdot v\,d\theta\backslash a|b\equiv c
\]
Applying the $\Gamma$ transformation:
\[
\Lambda = \pm\mp\sum_{i=0}^{\infty}\frac{\nu}{c\theta}\div3\quad
\Pi\subseteq\Phi\supseteq\Psi\subset\Upsilon\supset\Omega
\]
for some constant $c\not\in\emptyset$.

We know that one of $\gamma$ and $\delta$ is true. Applying a logical reduction:
\begin{eqnarray*}
\gamma\wedge\delta &\Longrightarrow& \gamma\wedge\delta\wedge\omega.\quad
\psi\simeq\sigma \\
&\Longrightarrow& \frac{\gamma\wedge\delta}{\omega'}\vee\neg\epsilon.\quad
\pi\gg\theta\ll\phi \\
&\Longrightarrow& \perp.\quad \omega\prec\varepsilon\succ\xi\preceq\zeta
\succeq\lambda
\end{eqnarray*}
It then must logically follow that $\mu$ reduces to:
\[
\ln\left[\lim_{z\rightarrow0}\left(1+\frac{1}{z}\right)^z\right]
+ \left(\sin^2(x) + \cos^2(x)\right)
= \sum_{n=0}^\infty \frac{\cosh(y)\sqrt{1-\tanh^2(y)}}{2^n}
\]
revealing that $f^2=g^2$.

\clearpage

\begingroup
\newcount\testcount \testcount=0
\font\myfont zmnex at 60pt
\noindent
\loop\ifnum\testcount<128 \hbox to 0.16\textwidth{\the\testcount: \myfont\vrule\char\testcount\vrule\hss}\hfil\advance\testcount by 1\repeat
\endgroup

\clearpage

\[
\sqrt{\sqrt{\sqrt{\sqrt{\sqrt{\sqrt{\sqrt{a}}}}}}}
\]
\def\delimtest#1#2{\[
\left#1   \left#1   \left#1   \left#1   \left#1   \left#1   \left#1
\left#1   \left#1   \left#1   \left#1   \left#1   \left#1   \left#1 a
\right#2^X\right#2^X\right#2^X\right#2^X\right#2^X\right#2^X\right#2^X
\right#2^X\right#2^X\right#2^X\right#2^X\right#2^X\right#2^X\right#2\]}
\delimtest()
\delimtest[]
\delimtest\langle\rangle
\delimtest\lfloor\rfloor
\delimtest\uparrow\downarrow
\delimtest\Uparrow\Downarrow
\delimtest\{\}


\end{document}
