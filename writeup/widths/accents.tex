\sectionname{Placement of Accents}

Another use of the measurement library is in determining the placement of
accents on the tops of characters. It is often not satisfactory to simply place
accents at the center of the character, in particular for characters with
ascenders (lowercase ``b,'' ``d'') and slanted or italic letters.

To measure the proper accent positioning, we draw a line at about 90\% of the
height of the character and find its intersections with that character's
outline. We then take the extreme left and extreme right intersection and
average them, taking that to be the approximate center of the character. The
accent is placed directly above that average point, skewed to compensate for the
italic slant.

This method works well for most characters, but does not work if the letter has
an open bowl or a left stem but no corresponding right (e.g., ``c,'' ``G'',
``E''). For those characters, we simply take the middle point of the letter and
place the accent directly above it (compensating for italic slant, again). For
the letter ``T,'' the measurement is taken at 50\% of the letter height, so the
accent is placed directly over the stem of the T.

For letters like ``b,'' ``d,'' ``h,'' and ``L,'' the accent is placed directly
over the stem of the character. It is questionable whether or not this is
desired behavior. On the one hand, the accent is skewed very far to one side, so
it is obviously off-center. On the other hand, placing the accent directly above
center would place it over blank space, which looks strange. The author's
decision was to place the accents off-center but not over blank space; if one
should desire the opposite effect, it would not be difficult to rewrite the
routine using the functions of the measurement library to do so.\looseness=-1,

\linespace

The resulting sidebearings and accent locations are not perfect. Many of the
characters are placed too tightly together (in particular, the right edge of the
lowercase c is too small), and some of the accents are placed incorrectly (in
particular, the accent is too far to the left on the \emph{f} in Palatino).
However, hand-tuning these values is not difficult, and with the automatically
measured values as a starting point, the job is made much easier for the end
user.
