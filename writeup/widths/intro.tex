\sectionname{Character Width \& Accent Placement}

Simply creating a set of mathematical symbols and then inserting the italic
letters as variables is not sufficient for high-quality math typesetting. The
italic letter variables must also be properly fitted and spaced, and accents
must be placed at their appropriate locations with those variables. The symbols
generated from the Computer Modern fonts already contain these necessary width
and placement adjustments, but the variables, taken from a text font without the
necessary mathematical placement dimensions, need to be measured for proper
placement and spacing.

Using our library of character measurement routines, we can implement a simple
algorithm for measuring the appropriate spacing and accent location for
characters. In the following sections we describe those algorithms; in the font
samples we show the accuracy of the measurements.
