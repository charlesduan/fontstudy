\sectionname{Character Widths and Math Fitting}

Characters in math mode require three parameters for proper fitting: the
position of the left edge of the character, the position of the right edge of
the character, and the position of the lower right edge where subscripts will be
placed. These positionings are known as ``sidebearings.''

These measurements could be carried out na\"\i vely by simply measuring the
extremities of characters and placing the sidebearings at those extremities.
This often does not result in good values. A character whose extremities are
very thin serifs, for example, should have its edges defined exactly at those
serif points. However, a character such as an ``o'' or an upright, sans serif
``\textsf{l},'' should have some extra space at the sides, or else adjacent
characters will appear too close.

Our algorithm for the measurement of sidebearings is as follows. First, the
absolute extremities of a character are measured. For the subscript positioning,
only the lower portion of the character is considered. The sidebearings are
required to be at least outside these absolute extremities.

Second, a field of $n$ horizontal lines is drawn across the character, and
the leftmost intersection of each line with the character is recorded. Those
intersections are sorted by x-coordinate, producing a list of sorted
x-coordinates $(x_1, x_2, x_3,\ldots)$. Some small unit of width $w$ and an
iteration maximum $N$ is chosen. The sidebearing is then constrained by the
following rule, where $s$ is the position of the left sidebearing:
\[
w-x_i \ge wi,\quad 1 \le i \le N
\]
The right sidebearing and subscript positions are positioned using an analogous
calculation.

If many points on the edge of the character are near the extreme left edge of
the character, then the sidebearing will be pushed further and further away from
the character. But if the extreme edge of the character is a thin serif, then
most of the points will already be far away from the sidebearing position, and
the sidebearing will end up right next to the serif.

This algorithm contains three parameters, $n$, $w$, and $N$. A higher value of
$n$ means greater accuracy in measurement but a longer computation time (since
more points on the character are tested). A higher value of $w$ means that the
sidebearing will tend to be pushed further away from the character. And the
value of $N$ can be used to enforce a ``maximum distance'' of the sidebearing
from the character, as the sidebearing will be no further than $wN$ from the
edge of the character.

This algorithm demonstrates the versatility of the simple functions provided by
the character measurement library. An even more complex scheme, also supported
by the library, would be to divide the character's edge into small segments and
find the extreme points on each segment, rather than simply looking at sampled
points on the edge of the character. This should provide an even more accurate
positioning.

Although it has not been implemented, a variation of this algorithm could be
used for determining kerning pairs between letters (the amount of horizontal
space to be added or removed between two characters for optimal fitting).
Although the results would not be perfect, it would give a good first impression
of kerning values between characters without requiring weeks of manual tuning.

