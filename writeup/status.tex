\sectionname{Current Status of the System}

The MathGen system is currently fully functional and in the process of being
made public. Currently the most important area of work is the user interface.

The program consists of seven Perl scripts and numerous auxiliary files that
perform the measurements and construct the font. Information on which font to
generate and special attributes of that font (e.g. whether or not the font is
sans serif) are read out of a special file that the user is expected to create.
Obviously this is an inconvenient system; a simple improvement would be an
interview-like program that would ask the user to locate the appropriate font
files and then create this configuration file automatically.

Additionally, although MathGen is intended to be run as-is, there are numerous
``advanced features'' sprinkled throughout the programs. The interface to these
disparate expert parameters should be consolidated into a single file. In fact,
there is already a main configuration file for constant parameter values not
usually altered by the end user; those advanced parameters will soon be moved
into that file and carefully documented.

Usage difficulty and current lack of documentation notwithstanding, MathGen has
proven through tests with numerous fonts to be a robust, successful system for
the generation of compatible math symbol fonts for a wide variety of text
typefaces. The ease of font generation and the aesthetic appeal of the output
will hopefully make its use popular and thus eliminate the common typographic
dissonance of incompatible math fonts that is all too common today.

As a testament to that success, the math fonts of this paper were not the
original Computer Modern math fonts; they were automatically generated by
MathGen.
