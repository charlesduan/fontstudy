\sectionname{How to Make a Font}

Assuming that all of the required programs are installed properly and in the
visible path, the generation of the math fonts proceeds simply as follows.

\paragraph{Install the text fonts.} Using fontinst, set up the desired font as a
text font. There are substantial references and manuals on doing this. Upon
doing this, \emph{take note of the names of these files:}
\begin{itemize}
\item The name of the \ac{TFM} file for the Roman (normal text) font
\item The name of the \ac{TFM} file for the Italic font
\item The name of the \ac{PFA/PFB} files for the Roman and Italic fonts
\end{itemize}
For example, the file names for Times New Roman would be \emph{ptmr8t} (roman
text font), \emph{ptmri8t} (italic text font), and no \ac{PFA/PFB} fonts, since
Times is built into standard printers.

\paragraph{Decompress the MathGen files.} Then enter the top-level directory
named \emph{mathgen}.

\paragraph{Bring over the necessary font files.} While in the \emph{mathgen}
directory, the \ac{TFM} files must be visible. To test this, type:
\begin{center}
kpsewhich [\ac{tfm} file]
\end{center}
If this prints out a "file not found" or other such error, then you may wish to
make sure that you have located the proper file and that it is installed
properly.

The \ac{PFA} files for the fonts must also be present in the \emph{mathgen}
directory or otherwise available.

\paragraph{Set up the \emph{fontparams.txt} file.} This file contains
information on the fonts to analyze and generate. It is a simple key-value file,
with keys separated from values by colons. The following keys are needed:
\begin{description}
\item[font\_file] The name of the \ac{pfa} file. If there are two or more files
necessary, this key may be entered as many times as necessary (but with only one
font per line). The file name must end in ``.pfa,''; otherwise MathGen will not
believe that it is a \ac{PFA} file.
\item[optical\_size] An optional suffix for the generated fonts. Usually this
should be left blank or not present; it is useful for distinguishing optical
variants of the font.
\item[output\_prefix] A short (3-letter) name for the math fonts. Usually they
should start with a ``z'' to distinguish the math fonts as ``user-defined.''
\item[roman\_name] The PostScript name for the roman variant of the font. This
can be found by reading the \ac{pfa} file. There should be a line like:
\begin{center}/FontName /[the name] def\end{center}
That name (without the leading slash) should be entered.
\item[italic\_name] The PostScript name for the italic variant. If, for some
reason, you want upright math characters, you can make this the same as the
\emph{roman\_name} value.
\item[roman\_tfm] The \ac{tfm} file for the roman variant of the font.
\item[italic\_tfm] The \ac{tfm} file for the italic variant of the font.
\item[sans\_serif] Is the font sans serif? Serifs are the small protrusions at
the terminal ends of letters such as ``A,'' ``I,'' and ''l.'' If serifs are
present on the font, enter ``yes''; otherwise enter ``no.'' Note that
\emph{slight} flares at the ends of letters do not count as serifs. (One way of
determining whether a font is sans serif is entering the value ``no''; if the
measurement routines generate errors, then change it to ``yes.'')
\item[square\_ends] Should the terminals of letters be squared or rounded off?
This affects the way that several of the characters look; it may be worthwhile
trying both ``yes'' and ``no.'' In general, sans serif fonts should have square
ends, and serif fonts should not.
\item[set\_param] Allows you to force a parameter to a certain value (which
should be in the units of the PostScript file, \emph{not} of the resulting \MF\
parameters). In general, this should not be used. One special case of this
directive is to set the \emph{no\_flare} parameter to 1; this will force the
\emph{flare} parameter to be equal to the stem width. Because the flares at the
ends of letters is an odd characteristic of fonts, it is very difficult to
measure reasonable values for the \emph{flare} parameter.
\end{description}
Comments may be included in the file, preceded by a hash mark (\#). Several
examples of valid fontparams.txt files are included.

\paragraph{Generate the PostScript file for measurements.} Running the script
\emph{./joinlibs.pl} will combine the specified font files, the font names, and
the PostScript routines in the \emph{letters} and \emph{libs} directories into a
single PostScript file \emph{output.ps}. This file may then be viewed with
GhostScript and even printed to a PostScript printer or distilled to PDF for
viewing.

\paragraph{Acquire the raw parameter measurements.} Running the script
\emph{./rawparams.pl} will run GhostScript on the \emph{output.ps} file and
extract the parameter values to a file \emph{out.txt}. If some of the
measurements appear wrong, that file may be modified.

It is possible that errors will come up in executing the routines. There are
several possible causes:
\begin{itemize}
\item The font is a sans serif font, but it was denoted in \emph{fontparams.txt}
as a serif font.
\item The font is irregular and has violated an assumption of the measurement
routines. MathGen is not designed for unusual display fonts; in such cases
automated parameter measurement may be impossible, so they must be done by hand.
\item There is a bug in the measurement routines. Rounding errors can cause
plenty of problems; in these cases you could try debugging the PostScript
libraries yourself or you can e-mail the font files to the author of the
program.
\end{itemize}
In any case, MathGen has been tested extensively on many different fonts, so it
is unlikely that errors should show up.

\paragraph{Make the new fonts.} The script \emph{makefonts.pl} will take the
parameter measurements and generate the fonts. This program will produce a great
deal of output, but as long as it terminates without errors, everything should
be ready.


