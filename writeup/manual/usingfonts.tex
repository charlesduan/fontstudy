\sectionname{Using the New Fonts}

Once all the scripts have been run, MathGen will produce a directory
\emph{output} that will contain all of the new fonts. Assuming the
\emph{output\_prefix} is \emph{xxx}, the files present will be:
\begin{description}
\item[mathxxx.map] The \ac{dvips} map file for the math fonts. Consult the
fontinst manual for its meaning and usage.
\item[mathxxx.sty] The \LaTeX\ style file for the math fonts. Including this
style file will instruct \LaTeX\ to use the new fonts for typesetting all math.
\item[*xxx.fd] The \LaTeX\ font definition files for the math fonts. Consult the
fontinst manual for their placement.
\item[*.pfa] The \ac{pfa} files for the new fonts.
\item[*.tfm] The \ac{tfm} font files.
\item[*.vf] The virtual font files, which combine the font's own italic letters
with the generated Greek letters to form a single math font.
\end{description}
For long-term use, these files should be moved to their appropriate directory
locations, as specified in the fontinst manuals. For quick testing, though, they
can be left in place.

A sample test file is placed in the \emph{mathgen/auxiliaries} directory, named
\emph{testfont.tex}. To use it, copy it into the \emph{output} directory and
change the first few lines to use the appropriate text font and the new math
font. The file should compile as normal (be sure to instruct dvips or pdflatex
to use the new map file).
