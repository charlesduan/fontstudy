\sectionname{Required Programs}

Several external programs are required to run the MathGen program.

\paragraph{Perl.} The main scripts of the program are written in Perl.

\paragraph{\TeX\ and \MF.} Naturally, the fonts are produced using \MF, and they are intended for use with \TeX. Additionally, the Computer Modern font sources are necessary. Any common installation of Te\TeX\ should suffice.

\paragraph{GhostScript.} The measurement routines are all written in
PostScript, so a PostScript interpreter is required to perform them.
GhostScript is a free interpreter; alternatively, one could use Acrobat
Distiller, or even a PostScript printer, to acquire the measurements (although
the current setup of the system makes GhostScript the simplest option).

\paragraph{Mftrace.} With the prevalence of electronic document distribution
today, it is more mandatory than optional that the output of MathGen be
compatible with electronic document systems such as the Adobe Portable Document
Format. Bitmap fonts, such as those produced by \MF, are highly unsatisfactory
for \ac{pdf} documents. Programs such as \TeX trace, mftrace, and MetaFog can
convert \MF output to vectorized Type~1 fonts; mftrace is, in the author's
opinion, the simplest free option available. Mftrace requires a program that
can trace bitmaps into vector graphics; the program \emph{potrace} is
recommended for producing the best output. Note that mftrace requires the Python
interpreter; one useful improvement would be to port mftrace to Perl for
language consistency.

\paragraph{Fontinst.} This program, which comes as part of any standard \TeX\
distribution, sets up the fonts for \LaTeX.

\paragraph{Type~1 fonts.} The fonts must be in Type~1, rather than TrueType or
OpenType, format. For other formats there are converters to Type~1 format (e.g.,
FontForge, ttf2pt1). Type~1 fonts may come in an \ac{ASCII} text format or a
binary format, usually named \ac{pfa} or \ac{pfb} depending on which. MathGen
requires the file to be in \ac{ASCII} format; there are common and simple
converters available (GhostScript includes one).
