\sectionname{Loops in Characters}\label{s:loops}

One of the larger changes to the font programs was the modification to small
loops in characters. Loops appear in the $\beta$, $\zeta$, and $\xi$ characters;
vertical loops occur in the $\gamma$ and $\omega$ characters.

\begin{figure}
\centering
\newgray{grey}{0.5}
\psset{linecolor=grey,linewidth=0.5}
\begin{tabular}{c}
\begin{pspicture}(0,0)(2,2)
\psecurve(0.25,12)(0.25,2)(1,0.5)(1.75,1)(1,1.5)(0.25,0)(0.25,-10)
\end{pspicture} \\
Open, round \\ loop
\end{tabular}
\begin{tabular}{c}
\begin{pspicture}(0,0)(2,2)
\psecurve(0.25,12)(0.25,2)(1,0.80)(1.75,1)(1,1.20)(0.25,0)(0.25,-10)
\end{pspicture} \\
Closed, round \\ loop
\end{tabular}
\begin{tabular}{c}
\begin{pspicture}(0,0)(2,2)
\psbezier(0.25,2)(0.5,1)(1,1)(2,1)
\psbezier(0.25,0)(0.5,1)(1,1)(2,1)
\end{pspicture} \\
Square loop \\ \hbox{}
\end{tabular}
\caption{Three different styles of loops used in the Computer Modern Greek
fonts. The square loop will be chosen if the \emph{square\_ends} parameter is
chosen. Otherwise, if the thickness of the loop strokes is less than the desired
loop width, an open, round loop will be made. If the strokes would be too thick,
a closed loop will be made.}
\end{figure}

In Computer Modern, the hairline stroke width is small enough that an entire
loop can be drawn, leaving some empty space in the middle of the loop. In other
fonts such as Helvetica this is not the case. Because the edges of the loop
cannot overlap (otherwise the stroke filling routine would cause an error), the
Computer Modern programs originally just thinned the stroke width until the
strokes of the loop no longer overlapped. This was unsatisfactory for monowidth
fonts like Helvetica; the unexpected truncation of the strokes caused those
letters with loops to look undesirably thin.

To solve this problem, an entirely new loop drawing routine was installed. If
the edges of the loop did not collide, then a loop was drawn as normal.
Otherwise, the loop was treated as just an ellipse to be filled in, and the
incoming strokes that would have made up the loop retained their widths and just
terminated within the filled ellipse.

Additionally, modifications were made for fonts with square ends rather than
round ones. In that case, loops were always closed so that the edges collided,
and a rectangle was drawn rather than an ellipse. This made the ``loops'' appear
stylistically much more consistent with other characters in the typeface.

As examples, the Helvetica and Palatino fonts use loops with square ends. The
Didot and Tiffany fonts have round, open loops, and New Century Schoolbook
demonstrates round, closed loops.
