\sectionname{Bugs in the Computer Modern Fonts}

During the process of reviewing the Computer Modern math fonts to make changes,
I noticed several errors. They have since been reported to the proper
authorities, and I am awaiting their response.

\begin{figure}
\centering
\leavevmode\raise 1cm\hbox{Incorrect}~\begin{pspicture}(0,-0.5)(2,2)
\pscustom{ \psarc(1,3){3}{-110}{-70} \psarcn(1,3){2}{-70}{-110}
    \closepath \fill[fillstyle=solid,fillcolor=lightgray] }
\psline(1,0)(1,1)
\qdisk(1,0.5){2pt}\uput[45](1,0.5){$c$}
\psline[linestyle=dashed](0,0.5)(2,0.5)
\uput[d](1,0){$r$}
\end{pspicture}
\quad
\begin{pspicture}(0,-0.5)(2,2)
\pscustom{ \psarc(1,3.5){3}{-110}{-70} \psarcn(1,3.5){2}{-70}{-110}
    \closepath \fill[fillstyle=solid,fillcolor=lightgray] }
\psline(1,0.5)(1,1.5)
\qdisk(1,1){2pt}\uput[45](1,1){$c$}
\psline[linestyle=dashed](0,0.5)(2,0.5)
\uput[d](1,0.5){$r$}
\end{pspicture}~\raise 1cm\hbox{Correct}
\caption{Example of the general bug found in Computer Modern. The bottom edge of
the gray stroke (point $r$) should line up with the dotted line. Instead, the
programs specify the center of that stroke (point $c$) to align there.}
\end{figure}

In \MF, a ``stroke'' can be drawn by defining the left and right edges around
points along that stroke and then tracing through those edges. The locations of
those points are usually defined by the extremities of those edges (e.g., the
left edge of this stroke should have this $x$-coordinate). In several places in
the program, the edge of a stroke is defined by the wrong side of the stroke.
For example, in the $\phi$ character, the bottom and top of the circular bowl
are placed with respect to the middle of the circular stroke rather than the
stroke's edge. The same occurs in the $\beta$ and $\delta$. Also, the left
upward stroke of the $\beta$ is placed with respect to its right edge, but the
code clearly implies that the placement should be with respect to the left edge.

\begin{figure}
\centering
\begin{pspicture}(-0.5,0)(5,5)
\pspolygon*[linecolor=lightgray](0,0)(1,0.7)(1,5)(-0.5,4.5)(-0.5,4)(0,4)
\pscustom{
\psbezier(5,5)(4.5,3)(2.5,0.5)(0.5,0)
\psbezier(0.5,1)(2,1.5)(3.5,3.5)(4,5)
\closepath
\fill[fillstyle=solid,fillcolor=lightgray]
}
\pspolygon(0,0)(1,0.7)(1,5)(-0.5,4.5)(-0.5,4)(0,4)
\qdisk(1,0.7){2pt}\uput[20](1,0.7){$c$}
\qdisk(0,0){2pt}\uput[135](0,0){$e$}
\end{pspicture}
\caption{Example of the bug with the $\nu$ character in Computer Modern. The
error is with the small notch in the lower left corner of the character. Note
that, because the diagonal stroke is curved but $\overline{ec}$ is straight,
there is no satisfactory location of $c$.}
\end{figure}

Finally, there is a bug that I haven't found a definitive way to fix. The
character $\nu$ is made up of a vertical stem and a diagonal stroke. The bottom
of the diagonal stroke is cut off so that it does not extend below the diagonal
stroke, but it is cut off too much, so in certain conditions an unsightly
``notch'' will appear at the base of the character.

The reason that none of these bugs are problematic in the actual incarnation of
the Computer Modern fonts is that the thin stroke widths are so small that the
errors are barely visible. For fonts where the thin strokes are not so thin,
such as sans serif fonts like Helvetica, these problems become much more obvious
and troubling without correction.
