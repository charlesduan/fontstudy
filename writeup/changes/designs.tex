\sectionname{Changes to Character Designs}

In addition to many overarching changes to the general implementation of the
Computer Modern character programs, a few specific design changes were made.
These changes allowed for better character appearance under different
parameterizations or simply better appearance in the opinion of the author.

Under the second category of changes, the left stem of the $\beta$ and the right
upward stem of the $\gamma$ were thickened. In high-contrast fonts, these
characters tended to appear too light in comparison with the rest of the text.
Additionally, it seems that someone (probably the American Mathematical Society)
decided to slightly thin even the thicker areas of the $\beta$ in the current
programs; these areas were thickened to Knuth's original values in
\emph{Computers~\& Typesetting}, Vol.~E\@. Also, the thickness of the $\phi$
character was increased slightly. The $\phi$ should appear much like the letter
``o,'' but the cross-stroke would make it look too dark. As a result, Knuth
counteracted the excess darkness by widening the character and thinning the
thicker edges. It is the author's opinion that only one of these changes is
necessary, so the width of the character was left untouched but the extra
thinning was undone.

Three characters were changed to improve their robustness under parametrization.
The simplest was the $\kappa$ character, whose upper diagonal was modified in
two ways. First, in fonts with square ends (Section~\ref{s:squareends}), the
upper diagonal was made straight without a flare at the end. Second, the stroke
was programmed in such a way that when the diagonal was relatively thick, the
middle of the stroke might become too thin. This error was fixed by defining the
control points on that stroke more robustly.

In the $\Upsilon$ character, the thickness of the two branches extending out of
the central stem was originally defined as 3/5 the thickness of the central
stem. For monowidth fonts such as Courier, where all of the strokes have the
same thickness, this caused those two strokes to appear too thin compared to the
rest of the text. This was fixed by redefining the branch width as 3/5 of the
way between the font's thinnest stroke width and the central stem's width.

\begin{figure}[b!]
\linespace
\begin{center}
\begin{pspicture}(0,0)(5,5)
\pscustom{
\psline(4,5)(0,5)(3,2)
\psarcn(1,0){2.8284271247}{45}{0}
\moveto(0,0)
\psbezier(0,0)(0,1.5)(1.5,2)(3,2)
\stroke[linecolor=lightgray,linewidth=0.7cm]
\newpath
\psline(4,5)(0,5)(3,2)
\pswedge(1,0){2.8284271247}{0}{45}
}
\qdisk(4,5){2pt}\uput[r](4,5){1}
\qdisk(0,5){2pt}\uput[225](0,5){2, 3}
\qdisk(3,2){2pt}\uput[45](3,2){4}
\psline(2.7,2.3)(2.4,2)(2.7,1.7)
\qdisk(3.8284271247,0){2pt}\uput[r](3.8284271247,0){5}
\psarc(1,0){15pt}{0}{45}\uput{18pt}[22.5](1,0){$\theta$}
\psarc(0,5){15pt}{-45}{0}\uput{18pt}[-22.5](0,5){$\phi=90^\circ-\theta$}
\end{pspicture}
\end{center}
\caption{Construction of the $\delta$ character. Points 1, 2, 3, and 5 are
fixed, and the $y$-coordinate of 4 is fixed. The $x$-coordinate of 4 is chosen
such that $\phi+\theta=90^\circ$, so that the line from 3 to 4 is tangent to the
circular arc between 4 and 5. The light gray line represents the outline of the
character.}
\label{f:delta}
\end{figure}

Finally, the $\delta$ character required significant revision. The upper hook of
the character is defined by five points: the rightmost end of the hook (point
1), the top of the hook (point 2), the leftmost point on the hook (where it is
traveling downward: point 3), the point at which the hook intersects the top of
the lower loop (point 4), and the vertical tangent point at the right side of
the loop (point 5). If the stroke widths at points 1, 2, and 3 are all about
equal, then the hook becomes too bulky and some of the curves are positioned
incorrectly.

Looking at other Greek fonts revealed that, for fonts with uniform stroke width,
the upper hook was not drawn as a curve but rather as a straight horizontal
stroke and then a diagonal line leading into the loop of the $\delta$. The
program for generating the character was then modified so, if the stroke width
at point 3 was close to that at point 1, the character would take this shape. In
the case of a square-end font, the upper horizontal stroke was squared off;
otherwise it curved toward the diagonal stroke.

The next question, then, became how to ensure that the diagonal stroke entered
into the loop. Experimentation showed that it was most aesthetically pleasing
for the diagonal stroke of the hook to be perfectly straight, and it was also
best for the part of the loop between points 4 and 5 to be as circular as
possible. Points 3 and 5 were fixed by the character's design already. So the
program was written to solve for the optimal location of point 4 such that, if a
line were drawn from point 3 to point 4 and a circle through points 4 and 5, the
angle of the line would be the same as the angle of the circle at point 4. The
construction is shown in Figure~\ref{f:delta}.

In general, this calculation chooses a point about 60\% of the way through the
width of the character. In a font with very small x-height, such as Futura, this
point is much closer to the right edge of the character; for a font with large
x-height, such as Avant Garde, this point is closer to the center of the
character's width. The generated $\delta$ characters for Futura and Avant Garde
both look acceptable, showing that this placement calculation is robust for
different parameterizations.
