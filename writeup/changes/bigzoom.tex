\sectionname{Scaling Large Characters}

Mathematics typesetting requires characters such as parentheses and summation
signs in sizes much larger than the normal text. Such large characters should
appear thicker than their normal-sized counterparts. How should the relative
thicknesses of such characters be determined?

The original Computer Modern programs increased the size of large characters by
adding ad-hoc dimensions that just happened to be the right sizes. A \emph{dw}
parameter is used as a small width increment in many of these characters. That
parameter is defined as the difference in width between the font's \emph{curve}
parameter (the thickness of the sides of o) and \emph{stem} parameter (thickness
of l). This relation is completely arbitrary, and with any font other than
Computer Modern it produces very strangely-sized large characters.\looseness=-1

Instead, MathGen provides new ``zoom'' parameters. The idea behind these
parameters is that the thickness of large characters should be proportional to
their size, but not directly proportional. A character that is twice the height
of another should be thicker but not twice as thick. The zoom parameters specify
a fraction of the size increase that should contribute to the increased
thickness. Say, for example, that one character has height $\alpha$ and another
has height $\beta$; let the zoom parameter be $\zeta$. Furthermore, assume that
some feature of the character at size $\alpha$ has width $\omega_\alpha$. Then
that feature for the character of size $\beta$ will have width:
\[
\omega_\beta = \omega_\alpha \cdot
\left( 1 + \zeta\left(\frac{\beta}{\alpha} - 1\right)\right)
\]
If $\zeta=0$, then the thicknesses of characters will be constant regardless of
size; if $\zeta=1$, then character thickness will scale linearly with size.

Two zoom parameters are provided. The first controls the increased thickness of
delimiters (parentheses, brackets, braces, etc.); this is set by default at
0.25. The second controls the increased thickness of display operators
(summation, integral, etc.); this is set at 0.55. Because these values cannot be
measured out of just the roman font (for which all characters are about the same
size), they must be specified by the user, although the default values seem to
work well in most cases. Note that the text-size operators (summation, integral)
are not zoomed at all, as they should appear compatible with the rest of the
text.
