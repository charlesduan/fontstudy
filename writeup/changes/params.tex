\sectionname{New Parameters Added}\label{s:params}

The Computer Modern fonts were built out of sixty-two parameters defining
various dimensions and qualities of the letters. Naturally, even the most
standard-looking text fonts do not conform with the constraints of these
parameters, so in developing compatible math fonts it was necessary to add in
new parameters.

\paragraph{Arm Thickness} The thickness of the central bar in the $\Pi$
character and the arms in the $\Sigma$ and $\Gamma$ characters were originally
equivalent to the thickness of serifs. This produced undesirably thin lines in
fonts like Times Roman, where the serifs are very thin but the arms are
substantially thicker. The thickness of the $\Pi$ bar was made equal to the
thickness of the ``T'' crossbar, and the arms were made the same thickness as
the upper arm in the letter ``F.''

\paragraph{Diagonal Serif Bracket} The \emph{bracket} parameter defines the
amount of curvature between serifs and stems. This parameter is usually measured
for a thick, vertical stem. When a serif is attached to a thin or diagonal stem,
such as in the letters A or $\Lambda$, the amount of curvature must be adjusted.
However, the adjustment is not parametrized, so for certain fonts the resulting
curvature can appear fairly ridiculous.

Luckily, the only serifs that show this problem are in the $\Lambda$ character,
so a separate \emph{bracket} parameter is measured based on the serifs of the
letter A. This provides a very accurate measurement, and the serifs are much
more compatible.

\begin{figure}
\centering
\newgray{grey}{0.5}
\psset{linewidth=0.5,linecolor=grey}
\begin{tabular}{c}
\begin{pspicture}(0,0)(3,4)
\psset{curvature=0.8 0.1 0}
\psccurve(1.5,0.25)(2.75,2)(1.5,3.75)(0.25,2)
\end{pspicture} \\
Low \emph{superness}
\end{tabular}
\begin{tabular}{c}
\begin{pspicture}(0,0)(3,4)
\psset{curvature=1.5 -1 0}
\psccurve(1.5,0.25)(2.75,2)(1.5,3.75)(0.25,2)
\end{pspicture} \\
High \emph{superness}
\end{tabular}
\caption{The \emph{superness} parameter. Low values will result in
diamond-shaped bowls; high values will create square-shaped bowls. The value
ranges from 0 to 1; a perfectly circular bowl is created with a \emph{superness}
values of $\sqrt{2}$.}
\label{f:superness}
\end{figure}

\paragraph{Superness of O} The \emph{superness} parameter determines how round
or square circular arcs will appear, as explained in Figure~\ref{f:superness}.
For the letter ``O,'' the superness value is often different from the rest of
the font, primarily because the letter is so round that it requires special
treatment. As a result, the superness of the O is measured separately from the
general superness value for the font, and that special value is used only for
the $\Theta$ character.

\paragraph{Capital Serif Jut} The amount by which serifs extend from the main
stem of uppercase Greek letters is encapsulated in Computer Modern by a single
parameter, the \emph{cap\_jut}. In most fonts, the serif jut varies depending on
the character's shape. As a result, for the uppercase Greek letters, four
different serif jut values are measured and applied to different characters. For
the $\Gamma$ character's bottom serif, the length is determined by the bottom
serif of the uppercase F. For the $\Lambda$ character the serif jut is
determined by the capital A. For $\Pi$ it is determined by H, and the others
($\Upsilon$, $\Phi$, $\Psi$) are determined by I.
