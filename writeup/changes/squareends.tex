\section{Square Stroke Ends}\label{s:squareends}

The Computer Modern fonts terminate many strokes with a circular cap. This does
not match well with fonts like Helvetica, in which letters always are cut off
squarely.

For many of these letters, the circular cap is drawn by a routine named
\emph{circ\_stroke}, so a first attempt at implementing square ends was to
simply modify this routine. This worked reasonably well, but the exact
positioning of the rounded end was often incorrect. In particular, rounded ends
of vertical strokes would often ``overshoot'' the baseline or the character's
intended height by a small amount to counteract the optical illusion that
strokes with curved ends appear shorter than strokes with square ends. As a
result, each stroke ending with a round end was manually repositioned and
changed to a square end. The results of this transformation can be seen in the
Palatino font as well as most of the sans serif fonts. For example, the vertical
strokes of the $\kappa$ and $\mu$ letters are normally rounded; in square-end
fonts the strokes are squared off and slightly lowered (in relation to the
baseline and character height).

More complicated were diagonal circular strokes, such as the shorter arm of the
$\lambda$. In this case it was necessary to calculate the effective width of the
stroke at a horizontal cutoff. If a stroke is to appear to have width $\omega$
at an angle $\theta$ from the horizontal, then the horizontal cutoff width is:
\[
\omega_{\mathrm{horizontal}} = \frac{\omega}{|\sin\theta|}
\]
This formula was used for the diagonal strokes of $\lambda$ and $\pi$ as well as
many of the math symbols.

Other changes were made to the designs of the fonts. For example, loops on
letters such as $\gamma$ and $\varepsilon$ were stylistically made square, as
discussed in Section~\ref{s:loops}. Also, the hook of the delta was modified in
certain cases to appear more square.

Finally, many of the math symbols originally terminated in rounded ends (e.g.
$+$, $\times$, $\approx$, $\in$). In particular, the equals and minus signs'
strokes were rounded. This was problematic because symbols such as the long
right arrows ($\longrightarrow$, $\Longrightarrow$) are made up of an equals
sign and an arrow pulled closely together. If the strokes making up those
symbols (defined by the \emph{rule\_thickness} parameter) was too large, then
the diameter of the circular tip might have exceeded the overlap between the
abutting symbols, and consequently the long arrows appeared to have small gaps.

To resolve this problem, the equals and minus signs were redrawn with square
ends. However, these two symbols then looked inconsistent compared to the other
math symbols, so all of them were redrawn with square ends. At normal text
sizes, this change is barely noticeable, but the change is significant at high
resolutions or large sizes.
