\sectionname{Introduction}

The digital age has brought great advances to the world of typography. Typefaces
are cheaply and conveniently available, produced in high quantity and high
quality. However, the bulk of available typefaces are oriented toward the
typesetting of plain text in roman characters. Authors with need for a wider
range of symbols, in particular mathematicians and scientists who regularly use
Greek letters and other symbols in written works, have only a limited set of
fonts with all the necessary symbols from which to choose.

In the recent past, the traditional solution was to mix and match: a
high-quality text font was chosen by the author, and the mathematical symbols
were drawn from whatever font was available. The most common example of this is
the proliferation of documents set in Times Roman, but using the Computer Modern
set of math symbols. This creates inconsistent typography: the symbols have
substantially different visual attributes compared with the text. In the case of
Times Roman and Computer Modern, the Times Roman characters are substantially
heavier, they tend to have less prominent serifs, and the capital letters are
shorter than the Computer Modern symbols. Such inconsistencies create visual
distractions and abberations, decreasing the overall appearance and readability
of the text.

The optimal solution would be for an author to choose whatever text font he/she
desired and then have a compatible set of math symbol fonts to use in
conjunction with the text. However, we are not all font designers, so it is
necessary for those compatible symbols to be \emph{automatically generated} by
inspection of the text font. This is the goal of MathGen: to automatically
inspect a text font and then produce a high-quality set of math symbol fonts
that are visually compatible with the original text font.
