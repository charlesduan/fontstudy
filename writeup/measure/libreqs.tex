\sectionname{Requirements for the Library}

The goal of the measurement library is to take a common digital representation
of a font and to perform programmable analyisis on it. In MathGen, the font
representation of choice is the PostScript Type~1 format. This format is chosen
because it is natively supported by the PostScript programming language, so it
is not necessary to write or acquire routines for interpreting the font. As a
natural consequence, the library and the routines written with it are all done
in the PostScript language.

The Type~1 font format represents characters as outlines, series of conjoined
curves and lines that define the edges of the character. The PostScript language
provides a method for traversing and processing this outline, so the measurement
routines must use this outline for inferring qualities of the characters. As a
result, the library must offer mechanisms for identifying certain points on the
outline, such as local extrema and intersections with other lines. Judicious
identification of such points and their locations should allow the calculation
of the necessary font parameters.
