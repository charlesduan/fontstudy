\sectionname{Parameter Measurements}

Using the functions provided by the measurement library we can measure many of
the parameters of the Computer Modern font. Here we will consider a few of these
measurement algorithms.

\begin{figure}[b!]
\centering
\vskip 6pt
\begin{pspicture}(0,0)(3,5)
\pspolygon[fillstyle=solid,fillcolor=lightgray]%
(0,0)(3,0)(3,.5)(2,.5)(2,5)(0,4.5)(0,4)(1,4.25)(1,.5)(0,.5)
%\pspolygon[fillstyle=solid,fillcolor=lightgray](0,0)(3,0)(3,.5)(2,.5)(2,4.5)
%(3,4.5)(3,5)(0,5)(0,4.5)(1,4.5)(1,.5)(0,.5)
\psline(0,2.5)(3,2.5)
\uput[l](0,2.5){$\alpha$}
\uput[r](3,2.5){$\beta$}
\qdisk(1,2.5){2pt}\uput[135](1,2.5){$\gamma$}
\qdisk(2,2.5){2pt}\uput[45](2,2.5){$\delta$}
\psline{<->}(1,2)(2,2)
\end{pspicture}
\caption{Measurement of the stem width from the lowercase letter l. The line
$\overline{\alpha\beta}$ is drawn halfway up the height of the letter, and
intersections $\gamma$ and $\delta$ are found. The stem width is the length
$\gamma\delta$.}
\end{figure}

The \emph{stem} parameter specifies the width of upright stems in lowercase
letters such as l, d, and t. That parameter's value can be easily measured by
drawing a horizontal line halfway through the letter l, finding the (hopefully
two) intersections between the line and the character outline, and finding the
distance between those two intersections.

\begin{figure}
\centering
\vskip 6pt
\begin{pspicture}(0,0)(3,5)
\pspolygon[fillstyle=solid,fillcolor=lightgray](0,0)(3,0)(3,.5)(2,.5)(2,4.5)
(3,4.5)(3,5)(0,5)(0,4.5)(1,4.5)(1,.5)(0,.5)
\psline(0,2.5)(3,2.5)
\uput[l](0,2.5){$\alpha$}
\uput[r](3,2.5){$\beta$}
\qdisk(2,2.5){2pt}\uput[45](2,2.5){$\gamma$}
\qdisk(3,.5){2pt}\uput[45](3,.5){$\delta$}
\psline{<->}(2,1.5)(3,1.5)
\psline[linestyle=dotted](3,1.5)(3,.5)
\end{pspicture}
\caption{Measurement of \emph{cap\_jut} from an uppercase I. The line
$\overline{\alpha\beta}$ is drawn halfway through the letter, and point $\gamma$
is found. Point $\delta$ is located as the point on the extreme right of the
letter. The jut, then, is calculated as $y_\delta-y_\gamma$.}
\end{figure}

The \emph{cap\_jut} parameter specifies the distance by which serifs extend
horizontally from the main character stem of uppercase letters. This is measured
by drawing a horizontal line halfway up the capital I, finding the rightmost
intersection of that line with the character outline, finding the rightmost
point of the entire outline, and then subtracting to get the overall horizontal
distance between those points.

The use of the uppercase I in measuring the \emph{cap\_jut} raises an
interesting question: why measure the jut of the uppercase I, rather than that
of, say, an M or E, for which the horizontal serif jut can be just as easily
measured. The answer is found by looking at the predominant use of that specific
parameter in the math font programs: the \emph{cap\_jut} parameter is used in
the uppercase Greek letters $\Gamma$, $\Lambda$, $\Pi$, $\Upsilon$, $\Phi$, and
$\Psi$. Of these, the serifs on $\Gamma$ and $\Lambda$ require special treatment
(as shown by experimentation), and the serifs on $\Upsilon$, $\Phi$, and $\Psi$
are most like those of an uppercase I. For the $\Pi$ character, the serifs
should actually look like those on a capital H, so a separate measurement is
taken and applied only for the serifs on the $\Pi$ character. The problem of the
capital serif jut is discussed further in Section~\ref{s:params}.

The \emph{thin\_join} parameter specifies the stroke thickness at the ``join''
locations between arches and upright stems on letters such as h, m, and n. Since
it is not a simple task to analytically find the distance across a stroke given
only the stroke's outline, we find the stroke's width by sampling many possible
intersections. First we trace a subpath consisting of the underside of the arch
of the letter. Next we find a point near the top of the join location (at the
notch at the upper left of the letter n, for example). Then a series of lines
are drawn starting from that point and at different angles toward the lower
right of the letter. For each line drawn, we find the intersection of the line
with the subpath formed above and thus the distance between that intersection
point and the join location point. The minimum distance is taken to be the
stroke thickness and reported as the \emph{thin\_join} value.

\begin{figure}
\vspace*{6pt}
\begin{center}
\begin{pspicture}(0,0)(7,4)
\pscustom{
\moveto(0,4)
\lineto(5,4)\lineto(7,0)\lineto(6,0)
\curveto(5.25,1.5)(3.9,2)(3,2)\lineto(0,2)
\fill[fillstyle=solid,fillcolor=lightgray]\stroke\newpath
\psset{linestyle=dashed}
\moveto(2,2)\lineto(6,0)\lineto(5,2)\closepath\stroke\newpath
\psset{linestyle=dashed}
\moveto(4,1)\lineto(5,2)\stroke\newpath
}
\qdisk(4,1){2pt}\uput[d](4,1){$m$}
\qdisk(5,2){2pt}\uput[u](5,2){$c$}
\qdisk(4.55,1.55){2pt}\uput[d](4.55,1.55){$b$}
\qdisk(2,2){2pt}\uput[u](2,2){$a$}
\qdisk(6,0){2pt}\uput[45](6,0){$t$}
\qdisk(7,0){2pt}\uput[45](7,0){$T$}
\qdisk(5,4){2pt}\uput[-135](5,4){$C$}
\end{pspicture}
\end{center}
\caption{Measurement of the \emph{beak\_darkness} parameter. The solid line
represents the outline of the beak serif; the dotted lines are constructed.}
\label{f:beak}
\end{figure}

Finally, the \emph{beak\_darkness} parameter, which measures the amount of
curvature under beak serifs like the one terminating the top stroke of the
letter F, provides an example of a relatively complex measurement routine.
The explanation will reference the points in Figure~\ref{f:beak}. The
measurement construction is based on the subroutine for drawing beaks, in
\emph{C \& T} Vol. E, page 376~\cite{cte}.

First, the points $T$, $C$, and $t$ are determined by identifying extrema on the
beak serif; this is done without great difficulty. Point $a$ is defined as the
point on the path whose $x$-coordinate is halfway between $T$ and the left end
of the beak. Point $c$ is defined such that $\overline{ac}$ is horizontal and
$\overline{ct}\mathbin\|\overline{CT}$. Point $m$ is then defined as the midpoint of
$\overline{at}$, and $b$ is the intersection of the character outline with line
segment $\overline{mc}$. The \emph{beak\_darkness} parameter, then, is:
\[
\mathit{beak\_darkness} = \frac{bc}{mc}
\]

These and many other routines are used to measure the values of parameters for
the generation of the math symbol fonts.
