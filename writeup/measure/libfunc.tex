\sectionname{Functions Provided by the Library}

The library provides functions for determining several properties of a given
character outline. The more important functions are described below.

\paragraph{Intersections} It is often useful to identify the intersection of a
character's outline with a straight line. A routine is provided to find all the
intersections between a line segment and the character, and the user of the
function may then pick the relevant intersection by sorting the list of
identified points.

\paragraph{Extremes} It is also useful to identify extreme points on the curve
(highest, lowest, furthest left or right). A general routine returns all local
extrema on the curve; the user can identify the proper one by sorting through
the list.

\paragraph{Subpaths} Often it is desirable to perform an operation on only a
small portion of the character outline rather than the entire outline. The
subpath operation allows the user to select a portion of the outline and then
perform operations on it alone. The subpath may be defined by a starting and
ending point; the user can also specify a subpath to end when the direction in
which the path is traveling meets a certain condition (e.g., the subpath should
stop before it starts traveling straight upward).

\paragraph{Point Operations} A general set of ``convenience functions'' are
provided for adding, subtracting, scaling, averaging, and otherwise manipulating
two-dimensional coordinates.

\paragraph{Line Operations} Functions are also given that allow the user to draw
various lines by specification, such as lines traveling through a point with a
given angle. Such operations are useful for constructing lines to be fed into
the path intersection routine above.

\paragraph{Iteration} Although this is not specific to the library (the
PostScript language provides iteration constructs), iteration is useful for
performing, say, several measurements and then taking the minimum value found.

The above operations are, in fact, sufficient to perform reasonably accurate
measurements of all the parameters for the Computer Modern programs; they also
allow us to perform other measurements such as character spacing and accent
placement, discussed later.
