\sectionname{Assumptions about Text Fonts}

Because MathGen attempts to distill the overall shape of a font into the
sixty-two parameters defined by the Computer Modern programs, we must make
several assumptions about the nature of letter shapes. These assumptions fall
into two categories: assumptions that when violated, will break the program or
otherwise cause erroneous measurements, and assumptions that, when violated,
will allow the program to run successfully but will produce strange-looking or
incompatible fonts.

In general, the more ``standard'' a font looks, the more successful MathGen will
be, where ``standard'' is defined as Computer Modern. So long as the overall
shapes of the letters are much like those of Computer Modern---serifs in the
right places, thick strokes and thin strokes of uniform widths and in the same
relative locations, character height uniformity, and so on---the measurements
will be accurate and the resulting fonts will look correct.

The following are the major assumptions about the font; namely, those that must
be true for MathGen to successfully run.

\paragraph{Serifs} Naturally, fonts may have serifs, but the serifs must occur
in the same places as they do in Computer Modern. The serifs may have a wide
variety of shapes, but they must not be too large (e.g., if the serifs at the
ends of the crossbar of a T should be less than 50\% of the character's height).
Additionally, serifs \emph{must} be present on certain letters such as I and l
(otherwise, the font is sans serif, and it should be denoted as so when running
MathGen). Finally, mere flares at the ends of strokes (e.g., in Optima) are not
considered serifs. Serifs must constitute a rapid change in stroke direction or
width.

\paragraph{Strokes} Letters should have no additional strokes beyond those that
appear in the letters of Computer Modern. In particular, large swashes, loops,
or unexpected splotches of ink (placed intentionally by the font designer, of
course) will disrupt the measurements. For example, the letter I is assumed to
have a single vertical stroke, and the letter O is assumed to consist of a
single, closed, circular stroke. Fonts such as Linotext (see the samples at
back) fail to meet this assumption.

\paragraph{Stroke Uniformity} Strokes are assumed to have relatively uniform
width and direction. For example, some display fonts, intending to imitate the
appearance of old typewriters, place ``pockmarks'' or small holes in the
otherwise straight stems of letters. These unexpected distortions will cause
MathGen to not properly interpret the overall shape of characters.

\paragraph{Straightness} Strokes that are normally straight should be as
straight as possible. In particular, the capital I should be perfectly vertical,
and the sides of the letters A and V should not be curved.

\linespace

The following assumptions must hold for the output to look acceptable.

\paragraph{Inter-Character Uniformity} The thickness of upright stems, curved
strokes, and thin hairlines should be uniform for different characters of the
font. Additionally, if two regions have the same thickness in the Computer
Modern font, then in general, those two regions should have the same thickness
in the text font being measured.

\paragraph{Italic and Roman Similarity} The italic and roman fonts should be as
similar as possible in terms of stroke widths, character heights and widths, and
overall appearance. Since the parameters are measured out of the roman font, the
resulting math characters will be as compatible as possible with the roman text;
if the italic is not compatible with the roman, then the italic will not be
compatible with the math fonts either.

\paragraph{Stroke Width Uniformity} MathGen does not (yet) provide for simple,
straight strokes of non-uniform width. For fonts where strokes tend to flare at
the ends (e.g., Optima), MathGen will be unable to capture this change in stroke
thickness.

\paragraph{Standard Serifs} The Computer Modern programs provide for two types
of serifs: horizontal serifs, which attach to vertical stems, and vertical beak
serifs, such as those at the ends of the arms of E, F, and L. Six parameters
define the horizontal serifs (ten in MathGen) and five define the beak serifs.
Serif designs that do not fall within the space of these parameters cannot be
perfectly duplicated by MathGen. For example, in Palatino, the beak serifs jut
inward toward the letter and are slightly curved. MathGen can capture the inward
jut but not the curvature of the serif.

\linespace

Although these assumptions may seem austere, practically all standard text fonts
will follow these rules; the exceptions are generally unusual display fonts in
which one would not often typeset mathematics. Additionally, MathGen has been
designed to be as robust as possible, correcting egregious dimensional errors
and handling unusual character shapes as gracefully, if not accurately, as
possible.
