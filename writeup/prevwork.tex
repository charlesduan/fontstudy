\sectionname{Previous Work}

The original concept of an adjustable font was developed by Donald Knuth in his
\MF\ program and the Computer Modern typefaces designed with
it~\cite{mfbook,cte}. In those programs, the Computer Modern font was defined in
terms of sixty-two changeable parameters, each of which specifies an attribute
of the font, such as the thickness of letter stems and the height of lowercase
letters (the x-height). In these fonts, Knuth implicitly makes two reasonable
assertions. First, he asserts that all fonts carry a basic ``shape'': certain
stems by convention will be thick and other stems thin, certain characters will
be the same height as others, and so on. For example, in the letter ``M,'' the
leftmost stem is always thinner than the rightmost stem. Second, the aspects of
character shapes that can change will change in uniform ways: if the thickness
of the stem of the lowercase letter ``l'' is increased, then the thickness of
the stems of other letters such as ``m,'' ``q,'' and ``k'' should increase as
well. 

These two assertions are not merely true for the fonts that can be generated
from the Computer Modern programs. In fact, observation of many text fonts
reveals that, in most ways, text fonts (that is, those designed for high
readability over long passages of text, as opposed to those designed to have
some unusual or idiosyncratic appearance) adhere to the general character shapes
and uniformities that those sixty-two parameters and the Computer Modern
programs define.

Because text fonts adhere to many of those general assumptions about character
design encapsulated in the Computer Modern fonts, it should seem reasonable
that, by measuring out the appropriate values for those sixty-two parameters,
one could use the Computer Modern font programs to generate new sets of
characters for any font. This idea was first put to use by Alan Hoenig in
\emph{MathKit}~\cite{mathkit}. In MathKit, the user would measure each of the
parameter values (by hand, with a ruler or on-screen calipers) and then execute
a series of scripts that would generate the new math symbols and then combine
them with italic letters to create a unified, compatible math font.

MathKit's approach will in fact produce good-quality math fonts with the
appropriate amount of work; in many situations the output of MathGen will be
practically the same as that from MathKit. There are several substantial
differences, though. First, MathKit requires the end user to perform the
measurements by hand. This is a difficult task, not because of the difficulty of
actually performing the measurements, but because the user does not know what to
measure. Although MathKit advertises that the user need not know \MF, the
meaning of the parameters is written only in the first chapter \emph{Computers
\& Typesetting}, Vol.\ E, and even there the parameters are not fully defined.
The only way to understand what exactly to measure is by reading through the
\MF\ programs to understand how the values are used. This requires not only
learning \MF\ but also spending many hours interpreting the programs, which, as
Knuth observes, are ``actual `optimized' code'' rather than
``straightforward\ldots textbook examples.'' It is the author's personal
experience that this is an unreasonable demand.

Second, even if the user is technically skilled enough and has sufficient time
to perform the measurements by hand, there are some that are simply very
difficult to actually perform. For instance, the \emph{beak\_darkness} parameter
measures the ``thickness of beaks,'' or the amount of ink used in filling the
portion between the horizontal arm and its terminal serif of letters like the
capital ``E.'' Measuring this parameter requires a tricky construction of a
triangle superimposed on the serif and then figuring out the fraction of the
triangle filled by the serif's internal curvature. A computer can easily do the
necessary construction, but it is fairly hard to do by hand. Even more difficult
are the \emph{superness} and \emph{superpull} parameters, which measure the rate
of curvature of circular letters. In MathKit the user is not even permitted to
change these and other such values, although experimentation has shown that even
small changes to these hard-to-measure numbers can produce substantially
different results.

For these reasons it seems that an \emph{automated} method of performing the
necessary measurements is strongly preferable to measurement by hand. The
question, then, becomes how to actually analyze a font's digitized form and
produce an accurate and complete set of measurements.
