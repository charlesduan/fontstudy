\sectionname{Palatino}

Palatino is a serif font with a strongly calligraphic design. It was originally
designed for display use, but it is now commonly used for text as well. Although
it is generally like a standard text font, it shows many unique display
characteristics that are difficult to duplicate automatically.

The overall appearance of the resulting math fonts is acceptable; the stem
weights and letter widths match properly. One minor problem is that the Palatino
italic letters are somewhat lighter than the upright roman letters; since the
Greek letters take after the uprights, they too are a bit darker than the italic
(although they match nicely with the roman text).

Many of Palatino's unique calligraphic effects, though, are not captured. The
angled interior of letters like ``o'' and ``p'' is not reflected, and the
terminals of letters (e.g., $k$ and $u$) are more angled while the Greek letters
are more horizontal. Additionally, although the inward-jutting beak of the ``F''
is present in the ``$\Lambda$'' character, the slight curve of it is not.

Palatino's math set was generated with the \emph{square\_ends} parameter on, so
that letters would not end in bulbs but rather with rigid cutoffs. This improved
the font compatibility significantly.
