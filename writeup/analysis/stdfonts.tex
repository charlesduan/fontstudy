\sectionname{Traditional Text Fonts}

Most standard text fonts follow certain conventions for typeface design: the use
of serifs at the ends of letters, stem thickness differences in predictable
places, consistent letter heights, and so forth. For these fonts it is not an
unusual style of design or anomalous-looking characters that differentiate them;
instead, differences in width, height, thickness, curvature, and other
measurable parameters make up the bulk of the uniqueness of such fonts. Because
these differences are so measurable and uniform, MathGen is adept at measuring
and mimicking them.

To ensure that MathGen is successful at producing aesthetically compatible math
fonts for many different text fonts, it was tested on a wide range of fonts.
Table~\ref{t:textfonts} lists some of the unique qualities of each of those
fonts.

\begin{table*}
\begin{center}
\begin{tabular}{rl}
Bookman: & wide, large x-height, heavy stems \\
Courier: & monospaced, no stem thickness contrast \\
Didot: & very high stem contrast, strong vertical stress \\
Adobe Garamond: & low stem contrast, small x-height, narrow \\
New Century Schoolbook: & heavy stems, thick serifs \\
Times Roman: & high stem contrast, thin serifs \\
Utopia: & squarish curved letters (high superness)
\end{tabular}
\end{center}
\caption{Unusual qualities of each of the standard text fonts tested.}
\label{t:textfonts}
\end{table*}
