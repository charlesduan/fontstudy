\sectionname{Didot}

The font Didot is a stark example of the modern style of type design, with its
high stem width contrast, strong vertical stress, and unadorned serifs. As the
Computer Modern fonts are also drawn in the modern style, one would expect the
programs to be highly capable of creating matching fonts.

The generated math fonts are surprisingly accurate. One distinctive feature of
Didot is the rapid thinning of thick strokes into thin, observed in the
lowercase and uppercase ``O''; this thinning is captured in the \emph{superpull}
parameter. As a result, similar circular strokes such as in $\theta$ and
$\Theta$ display the same rapid stroke thinning. The prominent darkness of beak
serifs is also mimicked nicely.

One unfortunate characteristic of Didot is that the italic letters are
substantially thinner than the upright ones. This causes the Greek letters to
appear much too dark. Additionally, the font seems to have reported its italic
slant incorrectly (the slant is read directly from a coded parameter in the font
rather than measured).

The designer of Didot chose to make the \emph{rule\_thickness} parameter (which
governs the thickness of strokes in mathematical symbols) about equal to the
hairline thickness. It is questionable whether that is desirable, as the math
symbols become barely visible at that thin size.
