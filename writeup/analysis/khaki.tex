\sectionname{Khaki}

Khaki is a font designed to look as if it were painted with a watercolor brush
(an alternate version includes random splatters as well). Since this is a
display font, perfect results are not expected, and it is pleasantly surprising
to see that the generated math fonts are actually reasonable.

Khaki violates some of the stem darkness assumptions of the font generation
program. In particular, on the letter ``O,'' the right side is thin and the
bottom thick; this is the opposite of what one would expect with a text font.
Additionally, the letters have a degree of ``texture'' to them: the stems are
slightly wavy, to mimic a watercolor brush.

The resulting math fonts have very similar stem thicknesses: the wide stems are
as wide as the widest in the original and the thin stems are equally thin.
However, the designer of Khaki used an unusually wide size for the thicker stems
and prevented the font from becoming too dark by making many areas that would
traditionally be thick into thin areas. For example, the lowercase ``b'' has an
unusually thick stem, but the adjacent bowl is entirely very thin, unlike any
traditional text font. Consequently, the math fonts look extraordinarily dark,
being unaware of this change. Also, naturally, the generated fonts do not
capture the texture of the strokes.

The generated characters do not look \emph{too} strange; clearly they are at
least better than using any off-the-shelf math font such as Computer Modern.
Possibly in this case it would be best to use the Computer Modern font programs
as a starting point, changing them to reflect the odd usage of stem widths, in
order to produce a more acceptable-looking font.
