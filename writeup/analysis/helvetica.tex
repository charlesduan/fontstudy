\sectionname{Helvetica}

Helvetica is a ``gothic'' or ``grotesque'' sans serif font, due to its simple
but non-geometric appearance. It exhibits a very strong apparent stroke
regularity---but not actual stroke regularity; for example, the thickness of the
joins of the arches in the ``h,'' ``m,'', ``n,'' and ``u'' are, upon close
inspection, significantly thinner than the rest.

The optical uniformity of stroke widths is captured in most places and
unexpectedly exaggerated in others in the generated math fonts. For example, the
downward curve of the $\zeta$ and $\xi$ characters is slightly but noticeably
thinner than other strokes. This is based on the fact that the thickness of the
top of the lowercase ``o'' is less than that of the sides, but unlike that
letter, the $\zeta$ and $\xi$ characters ``enlarge'' much more slowly. However,
it doesn't look too disconcerting to me; in fact, I actually rather like it that
way.

Another point about those same two characters: in most hand-designed sans serif
fonts, the upper loop is represented as a straight bar, like the top of the
redesigned $\delta$, and the terminal hook generally does not turn upward. This
is another example where a general-purpose redesign of the Computer Modern math
fonts for sans serif compatibility would be in order. Possibly the revision of
the $\delta$ character represents a start of that process.

Finally, close inspection of the roman Helvetica alphabet shows that almost all
of the stroke cutoffs are horizontal or vertical. This is true for many of the
strokes in the generated math fonts, but not all of them. However, this doesn't
affect the compatibility of the fonts.
