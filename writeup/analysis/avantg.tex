\sectionname{Avant Garde}

Avant Garde is a ``geometric'' sans serif typeface, meaning that its letters are
drawn from simple shapes like circles and straight lines. It is characterized by
very strictly monowidth strokes, an unusually large width and x-height, and
widely open counters and bowls (parts of letters surrounded partially or
entirely by the letter, such as the inside of an ``n'' or ``p'').

The generated math font is easily able to capture the consistent stroke width
and large x-height and width. Additionally, the \emph{superness} parameter
allows for many of the circular letters, such as $\sigma$ and $\Theta$,  to
appear roughly circular as well. One would expect the $\phi$ character to be
circular as well, but it generated as wider on the horizontal axis. That is
unfortunate but expected, as $\phi$ should be wider than $o$ to make up for the
additional visual color from the cross-stroke.

The Computer Modern Greek letters were not designed out of simple geometric
shapes as Avant Garde is. As a result, although the generated letters are
visually compatible with Avant Garde, they look stylistically very different. In
a sense, Avant Garde is something of a display font, working under a certain
design constraint. If the use of geometric typefaces is common enough, it might
be worth designing a set of math metafonts that work under this constraint of
simple geometric shapes; that could be used to generate math fonts that are both
visually and stylistically compatible with a range of geometric typefaces.
