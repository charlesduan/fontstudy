\sectionname{Linotext}

Linotext is an old-English blackletter font. As a very non-traditional display
font, it exemplifies the limits of MathGen pushed beyond the extremes.

Practically every assumption about letter shape made by MathGen is violated by
Linotext, which made it an interesting experiment. The attempt to measure stem
width from the capital letter ``I'' was thwarted by the fact that Linotext's
``I'' has two stems. Even stranger, the non-circular ``o'' caused the
\emph{superness} measurement to be completely wrong, making all the circular
bowls in the lowercase Greek letters ($\alpha$, $\beta$, $\theta$) appear like
diamonds. The resulting font is in some ways reminiscent of Donald Knuth's
``funny'' font, in which he chose arbitrary parameter values to see what the
font would do.

It is hard to conceive of any parametrization of the Computer Modern font that
would actually look compatible with this font. It is clear, though, that
computerized measurement of the letter features would be nearly impossible
without some assistance from the human eye.
