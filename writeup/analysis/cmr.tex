\sectionname{Computer Modern}

As discussed before, it is instructive to compare generated math fonts based on
Computer Modern against the original Computer Modern fonts. In most respects,
the generated fonts are indistinguishable from the originals. The two obvious
distinctions, the thicker stems of the $\beta$ and $\gamma$ characters, are
stylistic changes made in response to surveys observing that the stems of those
two characters in the original font were too thin. Additionally, the large
display operators (summation, product) are somewhat lighter than in the original
fonts. This is a matter of taste for the user; the operators' darkness could be
increased by simply raising the appropriate zoom factor.

Under \emph{extremely} careful inspection on high-resolution output, there is
one other difference. Small corners, such as the small hooks on the terminals of
letters like $\eta$, $\iota$, and $\upsilon$ are sharper in the generated fonts
than in the originals. This is due to the fact that, in Computer Modern, the
corners of letters, in particular the corners of serifs, are sharp in the roman
font but rounded in the italic. Since the ``corner roundness'' parameters are
measured from the roman font rather than the italic, the generated math fonts
have sharp corners as well. Note that this sharpness only affects very thin
strokes, so it is barely noticeable at normal sizes.

One possible way to correct this would be to measure corner diameter from the
italic font instead. However, italic fonts tend to have a less-defined
structure: the style of hooks, serifs, and other features differ a great deal
between italic fonts. As a result, it is difficult to make assumptions about the
shapes of characters for italic fonts, and so it is hard to find consistent ways
of measuring fine parameters such as corner diameter.
