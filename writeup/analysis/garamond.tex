\sectionname{Adobe Garamond}

Adobe Garamond is a very fine font designed after the typefaces of Claude
Garamond. It exhibits a liberal use of curves, a relatively low stem thickness
contrast, and a small x-height.

The italic face of Adobe Garamond is significantly narrower than the roman face,
so the Greek letters look too wide in comparison. Otherwise, the characters
blend well in terms of darkness and character shape. The generated fonts are not
able to capture the oblique interior of the ``O'' in the $\Theta$ character;
this would be a good place for improvement and additional parametrization.

One difficulty with Garamond is that, because it exhibits very long ascenders
and descenders \emph{and} it has a very sharp italic angle, the spacing is
fairly incorrect. In particular, letters with ascenders and descenders tend to
``push'' other characters away, creating unsatisfactory visual whitespace. This
could be solved by introducing kerning pairs into the math font, which would
require using the measurement libraries to determine an optimum kerning between
pairs of characters.
